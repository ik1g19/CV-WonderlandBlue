%%%%%%%%%%%%%%%%%%%%%%%%%%%%%%%%%%%%%%%%%
% Twenty Seconds Resume/CV
% LaTeX Template
% Version 1.1 (8/1/17)
%
% This template has been downloaded from:
% http://www.LaTeXTemplates.com
%
% Original author:

% Vel (vel@LaTeXTemplates.com)
%
% License:
% The MIT License (see included LICENSE file)
%
%%%%%%%%%%%%%%%%%%%%%%%%%%%%%%%%%%%%%%%%%

%----------------------------------------------------------------------------------------
%	PACKAGES AND OTHER DOCUMENT CONFIGURATIONS
%----------------------------------------------------------------------------------------

\documentclass[letterpaper]{twentysecondcv} % a4paper for A4














%----------------------------------------------------------------------------------------
%	 PERSONAL INFORMATION
%----------------------------------------------------------------------------------------

% If you don't need one or more of the below, just remove the content leaving the command, e.g. \cvnumberphone{}

%\profilepic{alice.jpeg} % Profile picture

\cvname{Isaac Klugman} % Your name

\cvdate{30 April 2001} % Date of birth
\cvaddress{134 Avenue Road} % Short address/location, use \newline if more than 1 line is required
\cvaddresstwo{Southampton SO14 6UA}
\cvnumberphone{07858248752} % Phone number
\cvsite{isaacklugman.com} % Personal website
\cvmail{isaacklugman@gmail.com} % Email address

%----------------------------------------------------------------------------------------











\begin{document}













%----------------------------------------------------------------------------------------
%	 EDUCATION
%----------------------------------------------------------------------------------------

%\setskilllist{Python, Java}
\newcommand{\myEducationTable}{
	\begin{tabular}{l}
		\educationitem{2019-2023}{MEng Computer Science}{University of Southampton}
		\educationitem{2017-2019}{A Level}{Bexhill College}
		\educationitem{2017}{GCSE}{Bexhill 6th Form}
	\end{tabular}
}
















%----------------------------------------------------------------------------------------
%	 QUALIFICATIONS
%----------------------------------------------------------------------------------------

\setmodulelist{Software Security, Web and Cloud Based Security, Computer Vision, Principles of Cyber Security, Biometrics, Distributed Systems and Networks, Intelligent Systems, Advanced Computer Networks, Intelligent Agents, Algorithmics, Advanced Databases, Algorithmic Game Theory, Data Management, Software Modelling and Design, Advanced Programming Language Concepts, Engineering Management and Law}













\makeprofile % Print the sidebar














%----------------------------------------------------------------------------------------
%	 PERSONAL STATEMENT
%----------------------------------------------------------------------------------------

\section*{Personal Statement}

I am a recent graduate of the University of Southampton and am excited to begin my career in industry. Throughout my studies, I have been passionate about problem solving and have developed interests in the cyber security and computer vision sectors, I am specifically proud of my work as a computer vision researcher for user position tracking systems at Audioscenic.  My web links provide a more detailed look at my university projects and practical experience. I am very passionate about teaching myself new things and will always try to find a way to pick up new skills - so a new and challenging project is always welcome. I am personable and a team player and have experience with working in several Agile/SCRUM project development life cycles through multiple University projects.













%----------------------------------------------------------------------------------------
%	 TECH SKILLS
%----------------------------------------------------------------------------------------


\section*{Technical Skills}

\setcvtags{Java, C++, C, C\#, Haskell, SQL, Python, BASH, HTML}
\cvtags













%----------------------------------------------------------------------------------------
%	 PROJECTS
%----------------------------------------------------------------------------------------





\section*{Projects}

\begin{tabular}{l} % Environment for a list with descriptions
	\projectitem{Audioscenic stuff}{University of Southampton}
	\projectitem{Online Notes Garden}{University of Southampton}
	\projectitem{GDP Project}{University of Southampton}
	%\twentyitem{<dates>}{<title>}{<location>}{<description>}
\end{tabular}











%----------------------------------------------------------------------------------------
%	 PUBLICATIONS
%----------------------------------------------------------------------------------------

\section*{Publications}

\begin{twentyshort} % Environment for a short list with no descriptions
	\twentyitemshort{2023}{Publication Title: 2023 International Conference on Engineering and Emerging Technologies (ICEET)\newline
		Article Title: Real-time 3D multi-person pose estimation using an omnidirectional camera and mmWave radars}
	%\twentyitemshort{<dates>}{<title/description>}
\end{twentyshort}















%----------------------------------------------------------------------------------------
%	 EXPERIENCE
%----------------------------------------------------------------------------------------

\section*{Experience}

\begin{twenty} % Environment for a list with descriptions
	\twentyitem{1900}{Alice in Wonderland-The Circra (1900's) Silent Film.}{Film}{The first Alice on film was over a hundred years ago.}
	\twentyitem{1933}{Alice in Wonderland 1933 version.}{Film}{This film stars Ethel griffies and Charlotte Henry. It was a box office flop when it was released.}
	\twentyitem{1951}{Disney Film.}{Film}{Walt Disney brings Lewis Carroll's fantasy story to life in this well done animated classic. Even though many elements from the book were dropped, such as the duchess with the baby pig and mock turtle, this version is without a doubt the most famous Alice adaption made.}
	%\twentyitem{<dates>}{<title>}{<location>}{<description>}
\end{twenty}

%----------------------------------------------------------------------------------------
%	 OTHER INFORMATION
%----------------------------------------------------------------------------------------

\section{Other information}

\subsection{Review}

test

%----------------------------------------------------------------------------------------
%	 SECOND PAGE EXAMPLE
%----------------------------------------------------------------------------------------

\newpage % Start a new page

\sidebartwo % Print the sidebar

\section{Other information}

\subsection{Review}

Alice approaches Wonderland as an anthropologist, but maintains a strong sense of noblesse oblige that comes with her class status. She has confidence in her social position, education, and the Victorian virtue of good manners. Alice has a feeling of entitlement, particularly when comparing herself to Mabel, whom she declares has a ``poky little house," and no toys. Additionally, she flaunts her limited information base with anyone who will listen and becomes increasingly obsessed with the importance of good manners as she deals with the rude creatures of Wonderland. Alice maintains a superior attitude and behaves with solicitous indulgence toward those she believes are less privileged.

\section{Other information}

\subsection{Review}

Alice approaches Wonderland as an anthropologist, but maintains a strong sense of noblesse oblige that comes with her class status. She has confidence in her social position, education, and the Victorian virtue of good manners. Alice has a feeling of entitlement, particularly when comparing herself to Mabel, whom she declares has a ``poky little house," and no toys. Additionally, she flaunts her limited information base with anyone who will listen and becomes increasingly obsessed with the importance of good manners as she deals with the rude creatures of Wonderland. Alice maintains a superior attitude and behaves with solicitous indulgence toward those she believes are less privileged.

%----------------------------------------------------------------------------------------

\end{document} 
